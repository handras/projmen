%% Based on a TeXnicCenter-Template by Gyorgy SZEIDL.
%%%%%%%%%%%%%%%%%%%%%%%%%%%%%%%%%%%%%%%%%%%%%%%%%%%%%%%%%%%%%

%------------------------------------------------------------
%
\documentclass[a4paper,12pt,leqno, notitlepage]{article}%
%Options -- Point size:  10pt (default), 11pt, 12pt
%        -- Paper size:  letterpaper (default), a4paper, a5paper, b5paper
%                        legalpaper, executivepaper
%        -- Orientation  (portrait is the default)
%                        landscape
%        -- Print size:  oneside (default), twoside
%        -- Quality      final(default), draft
%        -- Title page   notitlepage, titlepage(default)
%        -- Columns      onecolumn(default), twocolumn
%        -- Equation numbering (equation numbers on the right is the default)
%                        leqno
%        -- Displayed equations (centered is the default)
%                        fleqn (equations start at the same distance from the right side)
%        -- Open bibliography style (closed is the default)
%                        openbib
% For instance the command
%           \documentclass[a4paper,12pt,leqno]{article}
% ensures that the paper size is a4, the fonts are typeset at the size 12p
% and the equation numbers are on the left side
%
 
\setlength{\parindent}{2.2em}
\setlength{\parskip}{1.1em}
\renewcommand{\baselinestretch}{1.1}
\usepackage{amsmath}%
\usepackage{amsfonts}%
\usepackage{amssymb}%
\usepackage{graphicx}
\usepackage[utf8]{inputenc}
\usepackage[magyar]{babel}
%-------------------------------------------
\frenchspacing
\sloppy

% define references
\newcommand{\figref}[1]{\ref{fig:#1}.}
\renewcommand{\eqref}[1]{(\ref{eq:#1})}
\newcommand{\listref}[1]{\ref{listing:#1}.}
\newcommand{\sectref}[1]{section \ref{sect:#1}-\nameref{sect:#1}}
\newcommand{\tabref}[1]{\ref{tab:#1}.}

\begin{document}

\title{Agilis módszertan bevezetése}
\author{Horváth András \\ Széchenyi István Egyetem, Győr}
\date{2019. november 13.}
\maketitle

\begin{abstract}

Ez a dolgozat az IT projekmenedzsment tárgy teljesítéséhez készült. A dolgozatban összehasonlítom a kalsszikus és az agilis szoftver\-fejlesztési folya\-matot. Bemutatom ezek szereplőit, metódusait, alkalmazási területeit.

Ezután áttekintem a projektmenedzsment szereplőit és folyamatát. Célom bemutatni, hogy egy szoftverfejlsztéssel foglalkozó vállalatnál hogyan kerülhet bevezetésre az agilis módszertan. Legvégül kitérek arra kérdésre, hogy lehet-e átfedés vagy valamilyen kapcsolat a projekt-menedzseri feladatok és az agilis szoftver\-fejlesztés szerepkörei között.
\end{abstract}

\section{Szoftver\-fejlesztési módszerek}
\label{sec:s}

A dolgozat elején kettő, alapvetően különböző szoftver\-fejlesztési módszert tekintünk át. A két módszer között lényeges szemléletbeli különbségek vannak.

\subsection{Klasszikus szoftver\-fejlesztés}
\label{sec:Klasszikus}

Klasszikus szoftver\-fejlesztés alatt legtöbben a vízesés modellt értik. Ez a módszer régen nagyon elterjedt volt manapság viszont szinte elképzelhetetlen a használata összetett szoftver fejlesztése esetén. Ennek oka a módszer egyenes munkafolyamata és szigorúsága. A módszer fázisait az \figref{waterfall} ábra~\cite{waterfall_image} mutatja. Ezek az alábbiak:
\begin{description}
	\item[Szükségletek felmérése:] Ebben a fázisban a cél pontosan megérteni és dokumentálni mire van szüksége a vásárlónak.
	\item[Rendszer megtervezése:] A tervezés fázisában az előző fázisban elkészített dokumentumok alapján elő kell állítani a rendszer struktúráját.
	\item[Rendszer megvalósítása:] Ebben a fázisban történik a kódolás, a tervezett struktúra implementálása
	\item[Tesztelés:] Ebben a fázisban ellenőrizzük hogy amit implementáltunk az megegyezik a tervezettel, illetve hogy a vevő igényeit kielégíti-e.
	\item[Üzemeltetés:] Az üzemeltetés a szoftver karbantartása, utólagos fejlesztése ami a teljes ráfordítás 60\%-át is elérheti.\cite{waterfall}
\end{description}

A \cite{waterfall} forrás mindezen fázisok előtt megemlíti a megvalósíthatósági tanulmányt. Ennek célja eldönteni, hogy technológiailag lehetséges-e, illetve pénzügyileg mennyire éri meg a szoftverfejlesztésbe belefogni.

\begin{figure}[htb]
	\centering
		\includegraphics[width=0.85\textwidth]{images/waterfall.jpg}
	\caption{A vízesés modell fázisai \cite{waterfall_image}}
	\label{fig:waterfall}
\end{figure}

Ez a folyamat nagyon hasonlít az előadáson bemutatott álatalános projekt folyamattal. Abban szintúgy megtalálható a tervezés, megvalósítás és felügyelés lépései. A lényeges különbség az ott bemutatott folyamat és az itt leírt szofteverfejlesztési folyamat között, hogy ebben nincs lehetőség a fejlesztés megszakítására. Ez az üzleti életben nagy hátrányt jelent. Ebből adódik, hogy ezt a modellt manapság nem alkalmazzák nagy feladatokra. Ennek ellenére kisebb eszközök lefejlesztésére jó megközelítés.

Ennek a modellnek egy használható változatát kapjuk, ha egy iteratív ciklussá alakítjuk. Az elkészített terméket a megrendelő kipróbálja, használja. Az így szerzett tapasztalatok alapján új követelményeket fogalmaz meg és közli további szükségleteit. Ezután a ciklusban újra  a tervezés és a megvalósítás következik. Ez lehetőséget ad a hibák korai kijavítására. A korai módosítás, javítás lehetősége nagyon fontos egy öszetett alaklmazás fejlesztése során, mivel minél később történik annál nagyobb a költsége. \cite{fix_cost}


\subsection{Agilis szoftver\-fejlesztés}

Az 1990-as években a gyors változtatások és a felesleges munkavégzés elkerülése érdekében javasolták az agilis módszert. Az agilis szoftver\-fejlesztés valójában több fejlesztési folyamatot foglal magában. Ilyen folyamatok/módszerek például a \emph{Scrum}, \emph{eXtreme Programming} és a \emph{Lean}.\cite{agile} Ezek a módszerek valamilyen módon a fejlesztés hatékonyságának növelésére törekednek, így a projekt hamarabb és jobb minőségű szoftvert szállíthat.

A Scrum módszer lényege az iterációban és az inkrementális fejlesztésben rejlik. Nagyon hasonlóan a klasszikus vízesés modellhez, egy iterációban a 
\begin{itemize}
	\item követelmények összegyűjtése és megértése
	\item tervezés
	\item kódolás
	\item tesztelés és elfogadás
\end{itemize}
történik meg. Itt azonban mindez a változásra nyitottan történik meg. A fejlesztő csapat maga határozza meg, hogy a következő szoftver-szállítmányba adott idő alatt (ez a Sprint)  mit tud megvalósítani. A megrendelő a követelményet ízlése szerint priorizálhatja és a csapat az alapján fog sorban menni  a tervezés során. A szállítmányt a megrendelő kipróbálja és a következő tervezésre már vissza is tud jelezni és új követelményeket megfogalmazni.

A Scrum módszer definiálja, hogy egy fejlesztési projekt során milyen szereplők vesznek részt és milyen feladatokat látnak el:

\begin{itemize}
	\item Termék tulajdonos \emph{(Product Owner)}
	\item Scrum mester \emph{(Scum master)}
	\item Scrum csoport \emph{(Scrum team)}
\end{itemize}

Ezen szereplők feladatai az alábbiak:

A \textbf{termék tulajdonos} a megrendelők képviselője esetleg maga a megrendelő. Az ő felelőssége a befektetés megtérülésének maximalizálása. Ehhez a legfőbb eszköze, hogy a termék funkcionalitásait prioritizálja és ezáltal a csapattal ezeket készítteti el. Ha a csapat saját ötlettel áll elő, azt elfogadhatja vagy elutasíthatja. \cite{scrum_roles}

A \textbf{Scrum master} feladata a Scrum csoport és a termék tulajdonos támogatása, a helyes Scrum gyakorlatok elsajátításának támogatása. Ő nem a csapat vezetője vagy képviselője sem a projekt menedzsere. Ő egy befolyásos vezető, de nem közvetlen utasíásokkal vagy parancsokkal irányít. \cite{scrum_roles}

A \textbf{Scrum csapat} a szoftver fejlesztését végző személyekből áll. Összetételét tekintve több terület képviselőiből tevődik össze. A csapatnak önszerveződőnek és számonkérhetőnek kell lennie. A Sprint tervezésekor ők döntik el, hogy a szoftver-kiadásig hány funkciót tudnak megcsinálni és hogyan lehet ezt a legcélszerűbben elérni.\cite{scrum_roles}

Összességében az agilis módszertan előnye a rugalmasságában, a változásokra gyorsan reagáló képességében gyökerezik. Ezt a szemléletmódot minden szereplőnek el kell sajátítania a módszer sikeres alkalmazásához.
 
\section{A szoftver\-fejlesztés mint projekt}

Az előzőekben bemutatott metodológiák után most tekintsünk a szoftverfejlesztésre, mint projektre.

Minden projektre igaz kell hogy legyen, hogy időben és célkitűzésében, valamit személyi és pénzügyi erőforrásaiban elhatárolható más céloktól \cite{proj_fogalom}. Ez a kitétel egy szoftver fejlesztésére igaz lehet, de nem minden esetben. Pl.: egy operációs rendszer megalkotása rengeteg munka, és a ma leggyakrabban használtak már több évtizede fejlődnek. Operációs rendszer fejlesztésére általában szervezetet alapítanak, aminek ez a fő profilja. 

Egy mobil-játék lefejlesztése tipikusan sokkal hamarabb megvan, és többsége ha egyszer elkészült úgy is marad, befejezettnek van tekintve. Egy mobil-alkalmazást készítő szervezet életében több alkalmazás akár egyszerre párhuzamosan is készülhet. Ilyenkor egy alkalmazás elkészítése lehet egy önálló projekt, hiszen a cél elhatárolódik más tervektől és az időbeli  és pénzügyi erőforrások is elkülöníthetőek más tevékenységtől. Ennek az alkémazásnak van önálló megrendelője, aki a kiadásokat állja és a szervezet alkalmazottai közül saját csapatot alakítanak a fejlesztésre. A csapat tagjai akár több projekten is dolgozhatnak egyszerre, akár más-más feladatkört ellátva.

Jómagam a Knorr-Bremse DAS (Driver Assistance System) osztályán dolgozom. A osztályon vezetés-támogató rendszerek fejlesztése zajlik. Ilyen például az automatikus vészfékező rendszer (AEBS). Egy ilyen rendszer előállítása során nem csupán szoftver fejlesztést történik. A rendszer alapvető része a radar szenzor ami a detektálást végzi és amin általában a szoftver fut. Ennek a lefejlesztése szintén egy nagyobb projekt feladata. Ezen felül meg kell oldanai a hardver sorozat-gyártását és szállítását a járműgyártóknak.

Azonban a szoftver-fejlesztésre tekinthetünk egy önálló projektként egy nagyobb projekten belül. Ennek a projektnek a célja, hogy az adott hardverre egy adott feladatot megoldani képes szoftvert szállítson. Ebben a szeparációban a szoftverfejlesztés projekt megrendelője és így a vevője a rendszert előállító projekt. Abban a projektben döntik el, hogy a szoftvernek hogyan kell működnie, milyen platformon fut, és mikorra kell elkészülnie. 

\subsection{Egy agilis projekt szereplői}
\label{sec:AProjektSzereploi}

Azonosítsuk be, hogy az előbb felvázolt projektben a Scrum szerepkörök és a projektmenedzseri feladatok között milyen átfedés van. Az előadáson elhangzottak szerint haladva:

A projektmenedzser a projektet előkészíti és megtervezi. Maga mellé támogatókat keres, összeállítja a csapatot akik a megvalósításban részt fognak venni. A projekt során a megrendelő és a csapat közötti kapcsolattartást, egyetértést biztosítja. A Scrum metodológiában egyetlen lehetséges szerepe a \emph{product owner}, hiszen a \emph{PO} feladata a megerendelő képviselése a fejlesztő csapat felé. Ez azonban egyáltalán nem szükséges, erre szerepre választhat egy különálló személyt a \emph{projekt team}-en belül. A Scrum módszer szerint viszont fontos, hogy a projektmenedzser nem lehet \emph{Scrum master}.

A \emph{projekt team} a projekt megvalósításán dolgozó személyek csoportja. A Scrum módszerben bemutatott fejlesztő csapat egyértelműen ide esik. Ezen túl ide sorolható még a \emph{Scrum master} és akár a \emph{product owner} is, amennyiben erre a feladatra a menedzser egy külön személyt választ ki.

A projekt szponzor feladata, hogy a projektmenedzsert támogassa úgy szakmailag mint személyes ügyekben egyaránt. Egy jó szponzor ismeri a szervezet befolyásos személyeit és hogy "hogy mennek a dolgok". Ennél fogva el tud intézni olyan dolgokat amikre a projektmenedzsernek nincs lehetősége. \cite{proj_szponzor} Ennek a személynek jól állhat a \emph{Scrum master} szerepe. Ilyen például ha csapat jelzi felé, hogy gyorsan szükség van egy szerverre, akkor tudja, hogy az IT osztályon kinek kell szólni. Így amíg a projektmenedzser végigjárja a hivatalos folyamatot, a csapat addig is tud mit használni.

A megrendelőnek, a project \emph{steakholder}-ének a Scrumban a \emph{product owner} szerepkör jut, amennyiben személyesen tudja az érdekeit képviselni. Ha erre nincs ideje, akkor valakit felkér képviselőnek. 

\section{Az agilis módszer bevezetése}
\label{sec:AzAgilisModszerBevezetese}

Az eddig bemutatottak alapján felvázolom, hogy szerintem hogyan lehet egy nagyvállalati környezetben bevezetni az agilis módszertan szerinti szoftverfejlesztést.

Ha a klasszikus szemléletről egy agilis módszerre való átállásra egy projektként tekintünk, akkor erre igaznak kell lennie, hogy elhatárolódik más vállalati céloktól, tudjuk azonosítani a vevőt és erre a feladatra fel kell állítani egy projekt csapatot. 

Önmagában ezt a módszert nem lehet bevezetni, mivel ez egy "szabályrendszer", egy eszköz, amivel a szoftverfejlesztés hosszadalmas tervezését és dokumentálását lehet felgyorsítani. Tehát egy már folyamatban lévő szoftverfejlesztési projekt esetén van lehetőség átállásra. Természetesen egy újonnan kezdődő projekt esetén lehet egyből ezzel kezdeni. Arról, hogy a szoftver fejlesztése milyen módszerrel történik a projekt menedzsmentje hozza meg a döntést.

\bibliography{biblio}{}
\bibliographystyle{plain}

\end{document}
